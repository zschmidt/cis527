\documentclass{article}

%amsmath is used to define the "align" section
\usepackage{amsmath}

%I use geometry, because I'm a nut about margins
\usepackage{geometry}
 \geometry{
 a4paper,
 total={210mm,297mm},
 left=20mm,
 right=20mm,
 top=10mm,
 bottom=20mm,
 }

\begin{document}

%I hate the stock pound sign... so I fiddle with it here
\title{Assignment \raisebox{.22ex}{\large\#}1 \\
	CIS 427/527}
\author{Group 2}

\maketitle

%----------------------BEGIN PROBLEM 1----------------------

\section*{1}
Prove that for all natural numbers $n,\ n\leq 2^n$. 

\section*{Solution}

\begin{description}
\item[Base Case] Let $n=0$. Then $0 \leq 2^0 = 1$ holds.  
\item[Inductive Case] Inductive Hypothesis: $\forall n \in N, n \leq 2^n$.
\end{description}
\hspace{8em}Prove $n+1\leq 2^{n+1}$
\begin{align*}
n &\leq 2^n \\
2 n &\leq 2\cdot 2^n = 2^{n+1} \\
n + 1 &\leq n + n = 2n \leq 2^{n+1} \\
n + 1 &\leq 2^{n+1} \\
\end{align*}
Therefore, since P($n$)$\implies$ P($n+1$), $n\leq 2^n\ \forall n \in N$.

%----------------------BEGIN PROBLEM 2----------------------

\section*{2}
Consider the following form of induction.\\
\textit{strong induction:} To prove P($n$) is true for every natural number $n$, prove that for any natural number $m$, if P($i$) is true for all $i<m$, then P($m$) must also be true.\\\\
Prove using strong induction that the property P($n$) defined as\\\\
\hspace*{2em}if $n>1$ then there exist prime numbers $p_1,\cdot \cdot \cdot , p_k$ with $n = p_1 * p_2 * \cdot \cdot \cdot * p_k$.\\\\
holds for all $n$.

\section*{Solution}

\begin{description}
\item[Base Case] Prove P(2). This is true since 2 can be written as a product of primes, namely itself.
\item[Inductive Case] Prove P(2) $\land \ ...\ \land$ P($n$) $\implies$ P($n+1$) $\forall n > 1$. The inductive hypothesis states that $\forall m \in N,\ 2 \leq m \leq n,\ m$ can be written as the product of primes.
	\begin{itemize}
	\item $n+1$ is a prime, then like P(2) $n+1$ can be written as a product of itself
	\item $n+1$ is not a prime, then $\exists \ a, b$ such that $2 \leq a,b < n+1$ and $n+1 = a\cdot b$. By the inductive hypothesis, both $a,b$ can be written as the product of primes. Therefore, $n+1$ can be written as the product of primes.
	\end{itemize}
\end{description}
Hence, since P(2) $\land \ ...\ \land$ P($n$) $\implies$ P($n+1$), the property holds $\forall n \in N$.

\newpage
%----------------------BEGIN PROBLEM 3----------------------

\section*{3}
Prove that induction implies strong induction and vice-versa.


\section*{Solution}
If P(0) $\land \ ...\ \land$ P($n$) $\implies$ P($n+1$), then $\forall n \in N$, P($n$) $\implies$ P($n+1$) is certainly true (the first simply being the second fully written out). Thus, induction strong implies induction.\\
To show that induction implies strong induction, let $Q(n)$ be the property ``P holds from 0 to $n$'', then the induction axiom for $Q$ is:
\[
Q(0) \land [Q(n) \implies Q(n+1)] \implies \forall n\ Q(n)
\]
If we substitute the definition of $Q$ into the above, we get:
\[ 
P(0) \land [P(0)\land \ ...\ \land P(n) \implies P(0)\land \ ...\ \land P(n) \land P(n+1)] \implies [\forall n\ (P(0)\land \ ...\ \land P(n))]
\]
Which is logically equivalent to
\[ 
P(0) \land [P(0)\land \ ...\ \land P(n)] \implies P(n+1)
\]
Which is the strong induction axion. Thus, induction implies strong induction.\\\\
Therefore, induction and strong induction are equivalent.

%----------------------BEGIN PROBLEM 4----------------------

\section*{4}
Let us define a tree as follows:

\begin{itemize}
\item leaf is a tree
\item if t1 and t2 are trees then node (t1,t2) is a tree
\item nothing else is a tree
\end{itemize}

\noindent Define by recursion the following function rank (height, depth):

\begin{verbatim}
rank(leaf) = 1
rank(node(t1,t2)) = max(rank(t1),rank(t2))+1
\end{verbatim}

\noindent Using induction on the rank, prove that the number of leaves of any binary tree is at most one plus the number of internal nodes.\\

\noindent Prove the same by using structural induction.

\section*{Solution}
\textsc{Induction}
\begin{description}
\item[Base Case] A tree $t$ with one leaf has no internal nodes, therefore $l(t) \leq 1 + n(t) \to 1 \leq 1 + 0$.
\item[Inductive Case] Inductive Hypothesis: A tree with $i$ internal nodes and $l$ leaves satisfies $l \leq 1 + i$.\\
Assume trees $t,m$ (with \texttt{rank}$(t) = n$, \texttt{rank}$(m) \leq n$) has the property above. If we create a new tree $t'$ from $t$ and $m$ we have: 
\begin{align*}
\texttt{rank}(t') &\leq \textsc{Max}(\texttt{rank}(t), \texttt{rank}(m)) + 1\\
\texttt{rank}(t') &\leq \textsc{Max}(n, m) + 1\\
\texttt{rank}(t') &\leq n + 1\\
\end{align*}
Since \texttt{rank}$(t)$, \texttt{rank}$(m) < \texttt{rank}(t')$, the above property holds for $t'$ (from strong induction). Therefore, the inductive hypothesis holds $\forall n \in N$.
\end{description}

\newpage

\textsc{Structural Induction}
\begin{description}
\item[Base Case] A tree $t$ with one leaf has no internal nodes, therefore the hypothesis holds.
\item[Inductive Case] Let $l(t),n(t)$ denote the number of internal nodes and leaves in a tree $t$, respectively.\\
Inductive Hypothesis: $\forall t_1, t_2 \in T, l(t_1) \leq n(t_1) + 1$
\end{description}
\hspace{17em} $l(t_2) \leq n(t_2) + 1$\\\\
Prove: $l(\text{node}(t_1, t_2)) \leq n(\text{node}(t_1, t_2)) + 1$
\begin{align*}
l(\text{node}(t_1, t_2)) &= l(t_1) + l(t_2) \\
n(\text{node}(t_1, t_2)) &= n(t_1) + n(t_2) + 1\\
\text{By inductive hypothesis: }&\\
l(t_1) + l(t_2) & \leq n(t_1) + n(t_2) + 1\\
l(\text{node}(t_1, t_2)) & \leq n(\text{node}(t_1, t_2)) + 1\\
\end{align*}
Therefore, since $P(n) \implies P(n+1)$, the inductive hypothesis holds $\forall t \in T$.


%----------------------BEGIN PROBLEM 5----------------------

\section*{5}
Given the following inductive definition of an expression:

\begin{itemize}
\item 0 is an expression
\item 2 is an expression
\item if e1 and e2 are expressions then e1+e2 and e1*e2 are expressions
\item nothing else is an expression
\end{itemize}

\noindent where + is interpreted as addition and * as multiplication. Prove by structural induction that the value of every expression produced by this grammar is an even number.

\section*{Solution}
The base case has four cases
\begin{description}
\item[Base Case] Both expressions 0 and 2 are even numbers
\item[Recursive Case] This will be a proof by cases
	\begin{description}
	\item[e1+e2] Adding 0 or 2 to either 0 or 2 produces an even number, in addition to a valid expression. Thus, e1+e2 will always produce an even number.
	\item[e1*e2] Similarly, multiplying any expression by 0 or 2 will produce an even number, therefore e1*e2 will always produce an even number.
	\end{description}
\end{description}

%----------------------BEGIN PROBLEM 6----------------------

\section*{6}
Find the flaw with the following proof that $a^n = 1$ for all nonnegative integers $n$, whenever $a$ is a nonzero real number.

\begin{itemize}
\item BASE STEP: $a^0 = 1$ is true by the deinition of $a^0$.
\item INDUCTIVE STEP: Assume that $a^k = 1$ for all nonnegative integers $k$ with $k\leq n$. Then note:
\[ a^{n+1} = \frac{a^n * a^n}{a^{n-1}} = \frac{1*1}{1} = 1\]
\end{itemize}

\section*{Solution}
The error is in the inductive step, namely the expression $a^{n-1}$ which appears in the denominator. If $n=0$ (thus we are trying to show that $P(0) \to P(1)$), we have a negative integer $n$ for the exponent, which goes against the definition of problem.



\end{document}