\documentclass[10pt]{article}

%these two are needed for the tree drawings
\usepackage{graphicx,qtree}

%this package makes lists more compact
\usepackage{mdwlist}

%this is needed for \implies (and probably other stuff)
\usepackage{amsmath}

%latexsym is needed for \Box
\usepackage{latexsym}

%This section is needed for the valuation brackets
\usepackage{tikz}
\newcommand{\llbracket}{\
  \begin{tikzpicture}[scale=0.09,baseline=.3em]
  \draw (1.75,0) -- (0,0) -- (0,4) -- (1.75,4);
  \draw (1,0) -- (1,4);
  \end{tikzpicture}
  \
}
\newcommand{\rrbracket}{\
  \begin{tikzpicture}[scale=0.09,baseline=.3em]
  \draw (0,4) -- (1.75,4) -- (1.75,0) -- (0,0);
  \draw (.75,0) -- (.75,4);
  \end{tikzpicture}
  \
}

%This is me being a jerk about margins
\usepackage{geometry}
 \geometry{
 a4paper,
 total={210mm,297mm},
 left=20mm,
 right=20mm,
 top=10mm,
 bottom=20mm,
 }

\begin{document}

%I hate the stock pound sign... so I fiddle with it here
\title{Assignment \raisebox{.22ex}{\large\#}2 \\
	CIS 427/527}
\author{Group 2}

\maketitle

%----------------------BEGIN PROBLEM 1----------------------

\section*{1}
Since two of the men are telling the truth, we have a proposition of the form $P = (A \land B) \lor (A \land C) \lor (B \land C)$. 
\begin{itemize*}
\item If Adams is telling the truth, then Brown is lying $(A\implies \lnot B)$.
\item If Brown is telling the truth, then both Adams and Clark are lying $(B\implies (\lnot A \land \lnot C))$.
\item If Clark is telling the truth, then Brown is lying $(C\implies \lnot B)$.
\end{itemize*}

\noindent If Brown were telling the truth, then $P$ could never be satisfied, therefore Brown is lying, which makes him the killer.

%----------------------BEGIN PROBLEM 2----------------------

\section*{2}
\begin{description*}
	\item[2.] Show that $(( \to \notin PROP$
\end{description*}

\noindent Suppose $(( \to \in X$ and $X$ satisfies (i), (ii), (iii) of Definition 2.1.2. We claim that $Y = X\text{\textbackslash} \{ ((\to \}$ also satisfies (i), (ii), and (iii). 
\begin{description*}
\item[\hspace{2em}(i)] $\bot, p_i \in Y$, 
\item[\hspace{2em}(ii)] $\varphi , \psi \in Y$ and $(\varphi \Box \psi) \neq ((\to $, it is clear that $(\varphi \Box \psi) \in Y $
\item[\hspace{2em}(iii)] Similarly, $\varphi \in Y$ and $(\lnot \varphi ) \neq ((\to $, so $(\lnot \varphi ) \in Y $
\end{description*}
Therefore $X$ is not the smallest set satisfying (i), (ii), and (iii), so $((\to $ cannot belong to $PROP$.

\begin{description*}
	\item[7. (a)] Determine the trees of the proposition in Exercise 1


$(\lnot p_2 \to (p_3 \lor (p_1 \leftrightarrow p_2))) \land \lnot p_3$

\Tree [.$\land$ [.$\to$ [.$\lnot$ [.$p_2$ ]] [.$\lor$ [.$p_3$ ] [.$\leftrightarrow$ [.$p_1$ ] [.$p_2$ ]]]] [.$\lnot$ [.$p_3$ ]]]


$(p_7 \to \lnot \bot ) \leftrightarrow ((p_4 \land \lnot p_2 ) \to p_1$

\Tree [.$\leftrightarrow$ [.$\to$ [.$p_7$ ] [.$\lnot$ [.$\bot$ ]]] [.$\to$ [.$\land$ [.$p_4$ ] [.$\lnot$ [.$p_2$ ]]] [.$p_1$ ]]]


$(((p_1 \to p_2) \to p_1) \to p_2 \to p_1$

\Tree [.$\to$ [.$\to$ [.$\to$ [.$\to$ [.$p_1$ ] [.$p_2$ ]] [.$p_1$ ]] [.$p_2$ ]] [.$p_1$ ]]


  \item[\hspace{1.2em}(b)] Determine the propositions with the trees given

$\lnot \lnot \lnot \bot$ \\

$(p_0 \to \bot) \to ((p_0 \leftrightarrow p_1) \land p_5)$ \\

$\lnot ((\lnot p_1) \to (\lnot p_1))$

\end{description*}


\begin{description*}
	\item[9.] Show that a proposition with $n$ connectives has at most $2n + 1$ subformulas

  \begin{description*}
  \item[Base Case] A proposition $P$ with zero connectives has $\leq 2(0) + 1 = 1$ subforumla(s).
  \item[Inductive Case] Inductive Hypothesis: A proposition $\varphi$ with $n$ connectives has at most $2n + 1$ subformulas, where $s(\varphi)$ gives the number of subformulas (Def 2.1.3).\\
  Assume propositions $p,q$ with $n,m \leq n$ connectives resp., both of which have the property above. $s(p) \leq 2n+1$ and $s(q) \leq 2m+1$. Consider the proposition $r = (p \Box q)$. $r$ must have $n+m+1$ connectives. From the definition of s,
  \begin{align*}
  s(r) &= s(p) + s(q) + 1\\
  s(r) &= n + m + 1 \\
  s(r) &\leq 2n + 1
  \end{align*}


  \begin{align*}
	s(r) &= s(p) \cup s(q) \cup \{r\} \\
	|s(r)| &\leq |s(p)| + |s(q)| + |\{r\}| \\
	|s(r)| &\leq 2n + 1 + 2m + 1 + 1 \\
	|s(r)| &\leq 2(n+m+1)+1
  \end{align*}
  
  Since the desired property holds for $r$ the inductive hypothesis holds $\forall n \in N$.
  \end{description*}

\end{description*}

%----------------------BEGIN PROBLEM 3----------------------

\section*{3}

\textbf{(a)}

\begin{tabular}{ c || c | c }			
  $(\lnot \varphi \lor \psi) \iff (\psi \to \varphi)$ & $\varphi$ & $\psi$ \\
  \hline
  1 & 0 & 0 \\
  0 & 0 & 1 \\
  1 & 1 & 0 \\
  1 & 1 & 1 \\
  \hline  
\end{tabular}
Not a tautology.\\


\noindent \textbf{(f)}

\begin{tabular}{ c || c }			
  $(\varphi \lor \lnot \varphi)$ & $\varphi$ \\
  \hline
  1 & 0 \\
  1 & 1 \\
  \hline  
\end{tabular}
Is a tautology.\\

\noindent \textbf{(h)}

\begin{tabular}{ c || c }			
  $(\bot \to \varphi)$ & $\varphi$ \\
  \hline
  1 & 0 \\
  1 & 1 \\
  \hline  
\end{tabular}
Is a tautology.

%----------------------BEGIN PROBLEM 4----------------------

\section*{4}
\begin{description*}
  \item[(a)] $\varphi \models \varphi$

  By definition of semantic entailment, $\varphi \models \varphi$ iff for all $v: \llbracket \varphi \rrbracket _v = 1 \implies \llbracket \varphi \rrbracket _v = 1$. Since $\llbracket \varphi \rrbracket _v = 1 \implies \llbracket \varphi \rrbracket _v = 1$ is a tautology, $\varphi \models \varphi$ holds.


  \item[(b)] $\varphi \models \psi$ and $\psi \models \sigma \implies \varphi \models \sigma$

  By definition of semantic entailment, we have $\llbracket \varphi \rrbracket _v = 1\ \forall v \implies \llbracket \psi \rrbracket _v = 1$ and $\llbracket \psi \rrbracket _v = 1\ \forall v \implies \llbracket \sigma \rrbracket _v = 1$, by the transitive property, $\llbracket \varphi \rrbracket _v = 1\ \forall v \implies \llbracket \sigma \rrbracket _v = 1$.

  \item[(c)] $\models \varphi \to \psi \iff \varphi \models \psi$

  Since $\models \varphi \to \psi$, we know that $\llbracket \psi \rrbracket _v = 1\ \forall v$ (because if $\llbracket \psi \rrbracket _v = 0$ for some $v$, then $\varphi \to \psi$ is not a tautology). Therefore, we can conclude that $\llbracket \varphi \rrbracket _v = 1 \to \llbracket \psi \rrbracket _v = 1$, since anytime $\llbracket \varphi \rrbracket _v = 1$, $\llbracket \psi \rrbracket _v = 1$.


\end{description*}


%----------------------BEGIN PROBLEM 5----------------------

\section*{5}
\begin{description*}
	\item[1.] Show by algebraic means:
\end{description*}

\textbf{(i)} $\models (\varphi \implies \psi) \iff (\lnot \psi \implies \lnot \varphi)$ contraposition
\begin{align*}
	\varphi \implies \psi &\approx \lnot \varphi \vee \psi \text{Thm 2.3.4.b}\\
	\varphi \implies \psi &\approx \psi \vee \lnot \varphi \text{commutativity}\\
	\varphi \implies \psi &\approx \lnot \psi \implies \lnot \varphi \text{Thm 2.3.4.b}\\
\end{align*}

\textbf{(v)} $\models \lnot(\varphi \wedge \lnot \varphi)$
\begin{align*}
	\lnot(\varphi \wedge \lnot \varphi) &\approx \lnot \varphi \vee \lnot(\lnot \varphi) deMorgan's\\
	\lnot(\varphi \wedge \lnot \varphi) &\approx \lnot \varphi \vee \varphi Double Negation law\\
	\lnot(\varphi \wedge \lnot \varphi) &\approx \varphi \implies \varphi Thm 2.3.4.b \\
	\lnot(\varphi \wedge \lnot \varphi) &\approx \top (Not sure which rule for this) \\
\end{align*}

%----------------------BEGIN PROBLEM 6----------------------

\section*{6}
{\small \#} $= \lnot (\lnot (\lnot p \lor \lnot q) \lor \lnot (p \lor q))$

%----------------------BEGIN PROBLEM 7----------------------

\section*{7}
A conjunctive normal form is a tautology iff every clause is also a tautology.

%----------------------BEGIN PROBLEM 8----------------------

\section*{8}

%----------------------BEGIN PROBLEM 9----------------------

\section*{9}
\begin{description*}
  \item[$P\to Q$:] 

  True when $P$ is false; false when $P$ is true and $Q$ is false.

  \item[$P \lor Q \to P \land Q$:]

  True when $P\iff Q$; false otherwise.

  \item[$\lnot (P \lor Q \lor R)$:]

  True when $P,Q,R$ are all false; false when any $P,Q,R$ are true.

  \item[$\lnot (P \land Q) \land \lnot (Q \lor R) \land (P\lor R)$:]

  True when $P$ is true, $Q$ is false, $R$ is false; false when $P$ is false and $R$ is false.


\end{description*}


\end{document}
