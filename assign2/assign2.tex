\documentclass{article}

%amsmath is used to define the "align" section
\usepackage{amsmath}

%mdwlist makes more compact lists
\usepackage{mdwlist}

%latexsym is needed for \Box
\usepackage{latexsym}

%I use geometry, because I'm a nut about margins
\usepackage{geometry}
 \geometry{
 a4paper,
 total={210mm,297mm},
 left=20mm,
 right=20mm,
 top=10mm,
 bottom=20mm,
 }

\begin{document}

%I hate the stock pound sign... so I fiddle with it here
\title{Assignment \raisebox{.22ex}{\large\#}1 \\
	CIS 427/527}
\author{Group 2}

\maketitle

%----------------------BEGIN PROBLEM 1----------------------

\section*{1}
There are three suspects for a murder: Adams, Brown, and Clark. Adams says: “I didn’t do it. The victim was an old acquaintance of Brown’s. But Clark hated him.” Brown states ”I didn’t do it. I didn’t even know the guy. Besides I was out of town all that week.” Clark says “ I didn’t do it. I saw both Adams and Brown downtown with the victim that day; one of them must have done it.” Assume that the two innocent men are telling the truth, but that the guilty man might not be. Who did it?

\section*{Solution}
Since two of the men are telling the truth, we have a proposition of the form $P = (A \land B) \lor (A \land C) \lor (B \land C)$. 
\begin{itemize*}
\item If Adams is telling the truth, then Brown is lying $(A\implies \lnot B)$.
\item If Brown is telling the truth, then both Adams and Clark are lying $(B\implies (\lnot A \land \lnot C))$.
\item If Clark is telling the truth, then Brown is lying $(C\implies \lnot B)$.
\end{itemize*}

\noindent If Brown were telling the truth, then $P$ could never be satisfied, therefore Brown is lying, which makes him the killer.

%----------------------BEGIN PROBLEM 2----------------------

\section*{2}
\begin{description*}
	\item[2.] Show that $(( \to \notin PROP$
\end{description*}

\noindent Suppose $(( \to \in X$ and $X$ satisfies (i), (ii), (iii) of Definition 2.1.2. We claim that $Y = X\text{\textbackslash} \{ ((\to \}$ also satisfies (i), (ii), and (iii). 
\begin{description*}
\item[\hspace{2em}(i)] $\bot, p_i \in Y$, 
\item[\hspace{2em}(ii)] $\varphi , \psi \in Y$ and $(\varphi \Box \psi) \neq ((\to $, it is clear that $(\varphi \Box \psi) \in Y $
\item[\hspace{2em}(iii)]
\end{description*}
Therefore $X$ is not the smallest set satisfying (i), (ii), and (iii), so $((\to $ cannot belong to $PROP$.

\begin{description*}
	\item[7. (a)] Determine the trees of the proposition in Exercise 1
	\item[\hspace{1.2em}(b)] Determine the propositions with the following trees
\end{description*}

\begin{description*}
	\item[9.] Show that a proposition with $n$ connectives has at most $2n + 1$ subformulas
\end{description*}

%----------------------BEGIN PROBLEM 3----------------------

\section*{3}


%----------------------BEGIN PROBLEM 4----------------------

\section*{4}



%----------------------BEGIN PROBLEM 5----------------------

\section*{5}


%----------------------BEGIN PROBLEM 6----------------------

\section*{6}


%----------------------BEGIN PROBLEM 7----------------------

\section*{7}

%----------------------BEGIN PROBLEM 8----------------------

\section*{8}

%----------------------BEGIN PROBLEM 9----------------------

\section*{9}



\end{document}