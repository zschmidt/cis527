\documentclass[10pt]{article}

%these two are needed for tree drawings
\usepackage{graphicx,qtree}

%this package makes lists more compact
\usepackage{mdwlist}

%this package is to use the align environment
\usepackage{amsmath}

%We can get rid of the amsmath package with this definition
\renewcommand{\implies}{\Rightarrow \ }

%latexsym is needed for \Box
\usepackage{latexsym}

%This section is needed for the valuation brackets
\usepackage{tikz}
\newcommand{\llbracket}{\
  \begin{tikzpicture}[scale=0.09,baseline=.3em]
  \draw (1.75,0) -- (0,0) -- (0,4) -- (1.75,4);
  \draw (1,0) -- (1,4);
  \end{tikzpicture}
  \
}
\newcommand{\rrbracket}{\
  \begin{tikzpicture}[scale=0.09,baseline=.3em]
  \draw (0,4) -- (1.75,4) -- (1.75,0) -- (0,0);
  \draw (.75,0) -- (.75,4);
  \end{tikzpicture}
  \
}

%This package allows for the natural deduction proofs
%Can be found here: http://math.ucsd.edu/~sbuss/ResearchWeb/bussproofs/bussproofs.sty
\usepackage{bussproofs}

%This is me being a jerk about margins
\usepackage{geometry}
 \geometry{
 a4paper,
 total={210mm,297mm},
 left=20mm,
 right=20mm,
 top=10mm,
 bottom=20mm,
 }

\begin{document}

%I hate the stock pound sign... so I fiddle with it here
\title{Assignment \raisebox{.22ex}{\large\#}4 \\
	CIS 427/527}
\author{Group 2}

\maketitle

%----------------------BEGIN PROBLEM 1----------------------
\section*{1}

Complete the proof of soundness of propositional logic (given in val Dalen, Lemma 1.5.1) with the case of $\to _E$.

\section*{Solution}


%----------------------BEGIN PROBLEM 2
\section*{2}

Prove the soundness of the $\lor$ rules ($\lor I$ and $\lor E$).

\section*{Solution}

%----------------------BEGIN PROBLEM 3
\section*{3}

Do we have $\models (p \to q) \lor (q\to r)$?

\section*{Solution}


\begin{tabular}{ c | c | c | c | c || c }    
  p & q & r & $(p\to q)$ & $(q\to r)$ & $(p\to q)\lor (q\to r)$\\
  \hline
  0 & 0 & 0 & 1 & 1 & 1\\
  0 & 0 & 1 & 1 & 1 & 1\\
  0 & 1 & 1 & 1 & 0 & 1\\
  0 & 1 & 1 & 1 & 1 & 1\\
  1 & 0 & 1 & 0 & 1 & 1\\
  1 & 0 & 1 & 0 & 1 & 1\\
  1 & 1 & 1 & 1 & 0 & 1\\
  1 & 1 & 1 & 1 & 1 & 1\\
\end{tabular}

%----------------------BEGIN PROBLEM 4
\section*{4}

Do we have $ (q\to (p\lor (q\to p))) \lor \lnot(p\to q) \models p$?

\section*{Solution}

If both $p$ and $q$ are false, we have $q\to (p \lor (q\to p)) \lor \lnot (p \to q) = 1$ while $p=0$, breaking the semantic entailment.

%----------------------BEGIN PROBLEM 5
\section*{5}

Assuming the soundness and completeness of natural deduction for propositional logic, suppose that you need to show that $\phi$ is not a semantic consequence of $\phi _1, \phi _2, ..., \phi _n$, but that you are only allowed to base your argument on the use of natural deduction rules. Which judgement would you need to prove in order to guarantee that $\phi _1, \phi _2, ..., \phi _n \not\models \phi$? Do you need completeness and soundness for this to work out?

\section*{Solution}


%----------------------BEGIN PROBLEM 6
\section*{6}

Consider the following axiom based system, called Hilbert system:
\begin{align*}
& (\phi \to (\psi \to \phi))\\
& ((\phi \to (\psi \to \sigma)) \to ((\phi \to \psi ) \to (\phi \to \sigma)))\\
& ((\lnot \phi \to \lnot \psi ) \to ((\lnot \phi \to \psi ) \to \phi ))
\end{align*}
Combined with the Modus Ponens inference rule, which corresponds to the elimination rule of the implication connective.\\\\
Prove according to this system the judgement $\vdash \phi \to \phi$.

\section*{Solution}

\begin{prooftree}
\RightLabel{Let $\psi = \psi \to \varphi$ and $\sigma = \varphi$}
\AxiomC{$((\phi \to (\psi \to \sigma)) \to ((\phi \to \psi ) \to (\phi \to \sigma)))$}
\AxiomC{$(\phi \to (\psi \to \phi))$}
\BinaryInfC{$((\varphi \to (\psi \to \varphi )) \to (\varphi \to \varphi)))$}
\AxiomC{$(\phi \to (\psi \to \phi))$}
\BinaryInfC{$\varphi \to \varphi$}
\end{prooftree}

%----------------------BEGIN PROBLEM 7
\section*{7}

Consider classical logic given in the handout ``Natural deduction in sequent form'' in Figure 5. Prove the following judgements:\\
$\vdash \phi \lor \lnot \phi $ (This is called Law of Excluded Middle).\\
$((\phi \to \psi ) \to \phi )\to \phi $ (This is called Peirce's Law).\\

\section*{Solution}

%----------------------BEGIN PROBLEM 8
\section*{8}

Prove the following judgements:\\
$A\to B \to A$\\
$(A\to B \to C) \to (A\to B )\to (A\to C)$\\
$(A \land B \to C )\to(A\to B \to C)$\\
$(A\to B\to C)\to (A\land B \to C)$\\
Annotate each proof with lambda-terms.

\section*{Solution}

\begin{description*}
\item[] $A\to B \to A$

$(\lambda x: A.\ \lambda y: B.\ x)$


\item[] $(A\to B \to C) \to (A\to B )\to (A\to C)$

$(\lambda x: A\to B\to C.\ \lambda y: A\to B.\ \lambda z: A.\ x\ z\ (y\ z))$

\item[] $(A \land B \to C )\to(A\to B \to C)$

$(\lambda x: A\land B \to C .\ \lambda y: A.\ \lambda z: B.\ x (y*z))$

\item[] $(A\to B\to C)\to (A\land B \to C)$

$(\lambda x: A\to B\to C.\ \lambda y: A\land B.\ x(\text{fst } y)(\text{snd } y))$


\end{description*}

\end{document}
