\documentclass[10pt]{article}

%these two are needed for tree drawings
\usepackage{graphicx,qtree}

%this package makes lists more compact
\usepackage{mdwlist}

%this package is to use the align environment
\usepackage{amsmath}

%We can get rid of the amsmath package with this definition
\renewcommand{\implies}{\Rightarrow \ }

%latexsym is needed for \Box
\usepackage{latexsym}

%This section is needed for the valuation brackets
\usepackage{tikz}
\newcommand{\llbracket}{\
  \begin{tikzpicture}[scale=0.09,baseline=.3em]
  \draw (1.75,0) -- (0,0) -- (0,4) -- (1.75,4);
  \draw (1,0) -- (1,4);
  \end{tikzpicture}
  \
}
\newcommand{\rrbracket}{\
  \begin{tikzpicture}[scale=0.09,baseline=.3em]
  \draw (0,4) -- (1.75,4) -- (1.75,0) -- (0,0);
  \draw (.75,0) -- (.75,4);
  \end{tikzpicture}
  \
}

%This package allows for the natural deduction proofs
%Can be found here: http://math.ucsd.edu/~sbuss/ResearchWeb/bussproofs/bussproofs.sty
\usepackage{bussproofs}

%This is me being a jerk about margins
\usepackage{geometry}
 \geometry{
 a4paper,
 total={210mm,297mm},
 left=10mm,
 right=20mm,
 top=10mm,
 bottom=20mm,
 }

\begin{document}

%I hate the stock pound sign... so I fiddle with it here
\title{Assignment \raisebox{.22ex}{\large\#}5 \\
	CIS 427/527}
\author{Group 2}

\maketitle

%----------------------BEGIN PROBLEM 2.2.1----------------------
\section*{2.2.1}
Which of the following strings are formulas in predicate logic?
\section*{Solution}
\textbf{(a),(b),(f),(g)} are formulas.
\begin{description*}
\item[(c)] isn't, as $f(m)$ is a term.
\item[(d)] isn't, as $B$ is expecting two terms, yet $B(m,x)$ is a formula.
\item[(e)] isn't, as $B(m)$ doesn't have enough arguments.
\item[(h)] isn't, as $B(x)$ doesn't have enough arguments.
\end{description*}
%----------------------BEGIN PROBLEM 2.5.3
\section*{2.5.3}

\section*{Solution}

%----------------------BEGIN PROBLEM 2.5.11(a,b,c,e,f) 
\section*{2.5.11}

\section*{Solution}

%----------------------BEGIN PROBLEM 2.6.1
\section*{2.6.1}

\section*{Solution}

%----------------------BEGIN PROBLEM 2.6.2
\section*{2.6.2}

\section*{Solution}

%----------------------BEGIN PROBLEM 2.6.3 
\section*{2.6.3}

\section*{Solution}

%----------------------BEGIN PROBLEM 2.7.5
\section*{2.7.5}

\section*{Solution}


\end{document}
